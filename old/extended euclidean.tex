\documentclass[11pt, oneside]{article}   	% use "amsart" instead of "article" for AMSLaTeX format
\usepackage{geometry}                		% See geometry.pdf to learn the layout options. There are lots.
\geometry{letterpaper}                   		% ... or a4paper or a5paper or ... 
%\geometry{landscape}                		% Activate for for rotated page geometry
%\usepackage[parfill]{parskip}    		% Activate to begin paragraphs with an empty line rather than an indent
\usepackage{graphicx}				% Use pdf, png, jpg, or eps§ with pdflatex; use eps in DVI mode
								% TeX will automatically convert eps --> pdf in pdflatex		
\usepackage{amssymb}
\usepackage{amsmath}
\usepackage{parskip}

\graphicspath{{/Users/telliott_admin/Dropbox/Tex/png/}}

\title{Extended Euclidean algorithm}
%\author{The Author}
\date{}							% Activate to display a given date or no date

\begin{document}
\maketitle
\Large

We're interested in the Euclidean algorithm $gcd$ to find the greatest common divisor of two integers $a$ and $b$ (with $a > b$).

The method is to repeat these steps:  

(i) find $r = a$ mod $b$;  (ii) terminate if $r = 0$, returning $b$;  (iii) set $a = b$ and $b = r$.  So for example
\begin{verbatim}
60 = 4(13) + 8
13 = 1(8)  + 5
 8 = 1(5)  + 3
 5 = 1(3)  + 2
 3 = 1(2)  + 1
 2 = 2(1)  + 0
\end{verbatim}
The numbers on each line are

\begin{verbatim}
a = q(b) + r 
\end{verbatim}
where $q$ is the quotient (floor of $a/b$).  Skipping the last line, each line can be rearranged as $r = a - q(b)$

\begin{verbatim}
8 = 60 - 4(13)
5 = 13 - 1(8)
3 = 8 - 1(5)
2 = 5 - 1(3)
1 = 3 - 1(2)
\end{verbatim}

We can substitute for $2$ in the last line of the second part:
\[ 1 = 3 - 1(2) \]
\[ = 3 - 1 \ [ \ (5 - 1(3) \ ] \]
\[ = 2(3) - 1(5) \]

We can continue in this way, substituting $3 = 8 - 1(5)$
\[ = 2 \ [ \ 8 - 1(5) \ ] \ - 1(5) \]
\[ = 2(8) - 3(5) \]
and $5 = 13 - 1(8)$
\[ = 2(8) - 3 \ [ \ 13 - 1(8) \ ] \]
\[ = 5(8) - 3(13) \]
and finally $8 = 60 - 4(13)$
\[ = 5 \ [ \ 60 - 4(13) \ ] \ - 3(13) \]
\[ = 5(60) - 23(13) \]
Remember the $1$
\[ 1 = 5(60) - 23(13) \]

The implication of this last line is that if we're doing arithmetic mod $60$, then
\[ 5(60)  \text{ mod } 60 = 0 \]
so
\[ - 23(13)  \text{ mod } 60 = 1 \]
and since $-23 = 37  \text{ mod } 60$ we have finally
\[ 37(13)  \text{ mod } 60 = 1 \]
$37$ is the multiplicative inverse of $13$ mod $60$.

\subsection*{general case}

We imagine that we have a list like the one above from a run of the $gcd$ algorithm, where we have already thrown away the last line.
\begin{verbatim}
8 = 60 - 4(13)
5 = 13 - 1(8)
3 = 8  - 1(5)
2 = 5  - 1(3)
1 = 3  - 1(2)
\end{verbatim}

We number in reverse:
\[ r_5 = a_5 - q_5b_5 \]
\[ \dots \]
\[ r_1 = a_1 - q_1b_1  \]

For the case we're interested in, $r_1 = 1$ always, so let's just ignore the left-hand side for now.  We have the recurrence relationships.  Traversing the list from top down
\[ a_n = b_{n+1} , \ \ \ a_1 = b_2 \]
\[ b_n = r_{n+1}, \ \ \ b_1 = r_2  \]

The $q_n$ notation gets a bit awkward. I would like to suppress it to just use the digit, writing $n$ in place of $q_n$ at the appropriate times.

However, we also have a real integer 1, which I will write as $o$.

Round 1:  Start from 
\[ a_1 - q_1 b_1  \]

Round 2:  use the recurrence relation
\[ b_2 - q_1 r_2  \]
Substitute for $r_2$ 
\[ b_2 - q_1 \ [ \ a_2 - q_2 b_2 \ ] \]
Gather terms:
\[ - q_1a_2 + (1 +  q_1 q_2) b_2 \]
Suppress $q$ and write $o$ for integer $1$:
\[ - 1a_2 + (o +  12) b_2 \]

Round 3:  use the recurrence relation
\[ - 1b_3 + (o +  12) r_3 \]
Substitute for $r_3$
\[ - 1b_3 + (o +  12) \ [ \ a_3 - q_3 b_3 \ ] \]

Gather terms and suppress
\[  (o +  12)a_3 - (1 + 3 + 123) b_3 \]

Round 4:  use the recurrence relation
\[  (o +  12)b_4 - (1 + 3 + 123) r_4 \]
Substitute for $r_4$
\[  (o +  12)b_4 - (1 + 3 + 123) (a_4 - q_4 b_4) \]
Gather terms and suppress
\[  -(1 + 3 + 123)a_4 + (o + 12 + 14 + 34 + 1234)b_4  \]

Round 5:  use the recurrence relation
\[  -(1 + 3 + 123)b_5 + (o + 12 + 14 + 34 + 1234)r_5  \]

Substitute for $r_5$
\[  -(1 + 3 + 123)b_5 + (o + 12 + 14 + 34 + 1234) (a_5 - q_5 b_5)  \]
Gather terms 
\[ (o + 12 + 14 + 34 + 1234)a_5 - \dots  \]
\[  [  1  + 3 + 123 + 5 + 125 + 145 + 345  + 12345)  \ ] \ b_5 \]

I cannot find a pattern for this.  However, in the case of $(60,13)$, we should be done.  Recalling

\begin{verbatim}
60 = 4(13) + 8
13 = 1(8) + 5
 8 = 1(5) + 3
 5 = 1(3) + 2
 3 = 1(2) + 1
 2 = 2(1) + 0
\end{verbatim}

Since $a_5$ is equal to $60$ we need only the coefficient of $b_5$ i.e $13$.  But all of $q_1$ $q_4$ are $1$, while $q_5 = 4$ , and the encoded coefficient is
\[  1  + 3 + 123 + 5 + 125 + 145 + 345  + 12345  \]

Substituting
\[ 1 + 1 + 1 + 4 + 4 + 4 + 4 + 4 = 23 \]

And we must remember the leading minus sign, so we add $60 - 23 = 37$, which is correct.

\subsection*{consolidation}
Here is the list, for reference:
\begin{verbatim}
8 = 60 - 4(13)
5 = 13 - 1(8)
3 = 8  - 1(5)
2 = 5  - 1(3)
1 = 3  - 1(2)
\end{verbatim}

As another approach, consider that we have only two coefficients at each step, one for $a$ and one for $b$.  Why not just compute them and save those values as we go up the chain.

Round 1:  Start from
\[ a_1 - q_1(b_1)  \]
and then $q_1 = 1$
\[ a_1 - 1(b_1)  \]

Round 2:  use the recurrence relation
\[ b_2 - (1)(r_2)  \]
\[ b_2 - (1)(a_2 - q_2 b_2) \]
\[ (-1) a_2 + (1 + q_2 ) b_2 \]
and then $q_2 = 1$
\[ (-1) a_2 + (2) b_2 \]

Round 3:  use the recurrence relation
\[ (-1) b_3 + (2) r_3 \]
\[ (-1) b_3 + (2)(a_3 - q_3 b_3) \]
\[ (2)a_3 - (1 + 2q_3 )b_3 \]
 and then $q_3 = 1$
\[ (2)a_3 - (3)b_3 \]

Round 4:  use the recurrence relation
\[ (2)b_4 - (3)r_4 \]
\[ (2)b_4 - (3)(a_4 - q_4 b_4) \]
\[ -(3) a_4 + (2 + 3q_4) b_4 \]
and then $q_4 = 1$
\[ -(3) a_4 + (5) b_4 \]

Round 5:  use the recurrence relation
\[ -(3) b_4 + (5) r_5 \]
\[ -(3) b_4 + (5) (a_5 - q_5 b_5) \]
\[ (5) a_5 - (3 + 5q_5) b_5 \]
and then $q_5 = 4$
\[ (5) a_5 - (23) b_5 \]

We're almost done.  The coefficient of $b$ is $-23 = 37$ mod $60$.

We need to find a way to program this, given a list of the values of $q$.

\end{document}  