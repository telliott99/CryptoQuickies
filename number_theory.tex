\documentclass[11pt, oneside]{article} 
\usepackage{geometry}
\geometry{letterpaper} 
\usepackage{graphicx}
	
\usepackage{amssymb}
\usepackage{amsmath}
\usepackage{parskip}
\usepackage{color}
\usepackage{hyperref}

\graphicspath{{/Users/telliott_admin/Tex/png/}}
% \begin{center} \includegraphics [scale=0.4] {gauss3.png} \end{center}

\title{Number Theory}
\date{}

\begin{document}
\maketitle
\Large

This is a quick summary of concepts in Chapter 3 of 

\url{https://www.whitman.edu/mathematics/higher_math_online/}
\section*{Congruence}

For a positive integer $n$ and two integers $a$ and $b$, we say that $a$ and $b$ are congruent \emph{modulo} $n$, if they have the same remainder upon division by $n$, and we write
\[ a \equiv b \text{ mod } n \]

By this definition, if $a \equiv b$ mod $n$ we can write:
\[ a = qn + r \]
\[ b = q'n + r \]
and we see that the difference between $a$ and $b$ is evenly divided by $n$, with no remainder.
\[ a - b = (q - q')n \]

The converse is also true.  We write $n | a - b$ to mean that $n$ divides $a - b$ evenly.  If so, there is some $x$ such that
\[ a - b = xn \]
\[ a = b + xn \]
Since $b = q'n + r$
\[ a = q'n + r + xn \]
\[ a = (q' + x)n + r \]
Thus, the remainder when dividing $a/n$ is equal to $r$, as stated.

"There are a number of elementary properties which follow such as $a \equiv b$ implies $b \equiv a$.  The latter ones can can be summarized by saying that in any expression involving +,−,⋅ and positive integer exponents (that is, any "polynomial''), if individual terms are replaced by other terms that are congruent to them modulo $n$, the resulting expression is congruent to the original."

\subsection*{casting out nines}
In decimal notation, if the representation of $x$ is $d_n \ \dots d_2 \ d_1 \ d_0$, we mean that
\[ x = d_0 + d_1 \ 10^1 + d_2 \ 10^2 + \dots + d_n \ 10^n \]
All powers of $10 \equiv 1$ mod $9$ because, for example
\[ 10^2 \text{ mod } 9 = 10 \text{ mod } 9 * 10 \text{ mod } 9 = 1 * 1 \text{ mod } 9 = 1 \]
So
\[ x \text{ mod } 9 = (d_0 + d_1 + d_2 + \dots + d_n)  \text{ mod } 9 \]
Hence the left-hand side is zero only if $\sum d_i \text{ mod } 9 = 0$.

\subsection*{trick question}
Suppose $a = nq + r$.  What is the remainder upon dividing $a$ by $n$?  It is not necessarily $r$!

Example:  divide $20$ by $6$.  Well, $20 = 6 \cdot 2 + 8$ but $r > n$ and the correct answer is $2$. 

\subsection*{important result}
\[ ma \equiv mb \text{ mod } mn \ \ \  \iff \ \ \ a \equiv b \text{ mod } n \]

\section*{Notation}
For every integer $a$ we say that $[a] = [a']$ if and only if $a \equiv a'$.

Define $Z_n = \{ [0], \ [1], \ [2], \ \dots [n-1] \}$. For example:  $Z_4 = \{ [0],\ [1], \ [2], \ [3] \}$.  $Z_4$ is also equal to $\{−80],[25],[102],[−13]\}$, but this is just confusing.

This formal notation seems a bit too much to me.

Addition and multiplication tables for $Z_6$

\begin{verbatim}
+  0  1  2  3  4  5          x  0  1  2  3  4  5
0  0  1  2  3  4  5          0  0  0  0  0  0  0
1  1  2  3  4  5  0          1  0  1  2  3  4  5
2  2  3  4  5  0  1          2  0  2  4  0  2  4
3  3  4  5  0  1  2          3  0  3  0  3  0  3
4  4  5  0  1  2  3          4  0  4  2  0  4  2
5  5  0  1  2  3  4          5  0  5  4  3  2  1
\end{verbatim}

The usual laws apply for addition, subtraction and multiplication.  However, division is different.  To divide by $a$ instead, multiply by the multiplicative inverse $a^{-1}$.

The problem is that some values for $a$ don't have an inverse.  Another problem is that $ab = 0$ does not imply that either $a = 0$ or $b = 0$.

However, for $Z_p$ where $p$ is prime, it's all fine.  Here is that multiplication table for $Z_7$:

\begin{verbatim}
 x  0  1  2  3  4  5  6
 0  0  0  0  0  0  0  0
 1  0  1  2  3  4  5  6
 2  0  2  4  6  1  3  5
 3  0  3  6  2  5  1  4
 4  0  4  1  5  2  6  3  
 5  0  5  3  1  6  4  2 
 6  0  6  5  4  3  2  1
\end{verbatim}

Here, every value has a unique multiplicative inverse, and in fact, for every $a$ and $b$
\[ ax = b \]
has a unique solution.

Note something curious, if we talk about $Z_9$ and \emph {leave out those elements that do not have an inverse}, including zero, then everything works.

\begin{verbatim}
x  1  2  4  5  7  8
1  1  2  4  5  7  8
2  2  4  8  1  5  7
4  4  8  7  2  1  5
5  5  1  2  7  8  4
7  7  5  1  8  4  2        
8  8  7  5  4  2  1
\end{verbatim}

\section*{Euclidean algorithm}

We've already explored the Euclidean algorithm for finding the greatest common divisor or $gcd(a,b)$.  Suppose that $a > b$ (if not, switch them).  

Consider $a = 198$, $b = 168$.  The algorithm depends on the fact that if $m$ is a divisor of both $a$ and $b$
\[ mx = a \]
\[ my = b \]
then $m$ is also a common divisor of their difference.  
\[ a - b = m(x - y) \]
And if some multiple of $b$, say $cb$, can be subtracted from $a$ so that $cb = mcy$ then
\[ m(x - cy) = a - cb \]
$m$ is a common divisor of that difference.

One way of formulating the algorithm:  given $a$ and $b$, with $a > b$, find the largest $q$ so that $a = qb + r$.  If $r = 0$, the result is $b$.  Otherwise, if $r \ne 0$, set $a = b, b = r$ and repeat.

\[ a = q \cdot b + r \]
\[ 198 = 1 \cdot 168 + 30 \]
\[ 168 = 5 \cdot 30 + 18 \]
\[ 30 = 1 \cdot 18 + 12 \]
\[ 18 = 1 \cdot 12 + 6 \]
\[ 12 = 2 \cdot 6 + 0 \]

$gcd(198,168) = 6$.  Of course, if we factored these two numbers we would see right away that
\[ 198 = 2 \times 3^2 \times 11 \]
\[ 168 = 2^3 \times 3 \times 7 \]

\section*{linear combination}

It turns out that $m$ is a \emph{linear combination} of $a$ and $b$, meaning that there exist integers (one positive and one negative) such that:
\[ m = x \cdot a + y \cdot b \]

Rearrange the previous equations to be differences:
\[ r = (q)b - a \]
\[ 30 = 198 - (1)168 = a - b \]
\[ 18 = 168 - (5)30 \]
\[ 12 = 30 - (1)18 \]
\[ 6 = 18 - (1)12 \]
\[ 0 = 12 - (2)6 \]

Start with the first difference
\[ 30 = 198 - 168 = a - b \]
Go to the next difference
\[18 = 168 - (5)30 = b - 5(a-b) = - 5a + 6b
We can check this calculation: 
\[ -5 \cdot 198 + 6 \cdot 168 = -990 + 1008 = 18 \]

Continue
\[ 12 = 30 - (1)18 = (a - b) - 1(- 5a + 6b) = 6a - 7b \]
and then
\[ 6 = 18 - 12 = (-5a + 6b) - (6a - 7b) = -11a + 13b \]
Check again and see that this is correct.  

Thus we have expressed the $gcd$ as a linear combination of $a$ and $b$.

Now, for the cases we really care about, the $gcd$ will be equal to $1$, and we will end up with an equation like:
\[ 1 = x \cdot a + y \cdot b \]
and we will say that mod $a$
\[ 1 = y \cdot b \]


\end{document}  