\documentclass[11pt, oneside]{article} 
\usepackage{geometry}
\geometry{letterpaper} 
\usepackage{graphicx}
	
\usepackage{amssymb}
\usepackage{amsmath}
\usepackage{parskip}
\usepackage{color}
\usepackage{hyperref}

\graphicspath{{/Users/telliott_admin/Tex/png/}}
% \begin{center} \includegraphics [scale=0.4] {gauss3.png} \end{center}

\title{Chinese Remainder Theorem}
\date{}

\begin{document}
\maketitle
\Large

\url{https://www.cut-the-knot.org/blue/chinese.shtml}

\begin{quote}There are certain things whose number is unknown. Repeatedly divided by 3, the remainder is 2; by 5 the remainder is 3; and by 7 the remainder is 2. What will be the number?\end{quote}

- Sun Tsu Suan-Ching

Let $r$ and $s$ be positive integers which are relatively prime.  As a simple example let $r=4$ and $s=5$.  

Now, write the integers from $1$ to $r \times s$:
\begin{verbatim}
       |       |       |       |       | 
o o o o o o o o o o o o o o o o o o o o
1 2 3 4 5 6 7 8 9 0 1 2 3 4 5 6 7 8 9 0
         |         |         |         |
\end{verbatim}

Consider the numbers $n = [1-rs]$ 

Compute $n$ mod $r$ and mod $s$, and write the result as a pair or tuple.  For example:
\[ 10 \equiv (2,0), \ \ \ \ 18 \equiv (2,3) \]

Starting from $1$ and ending at $20$:
\begin{verbatim}
(1,1),(2,2),(3,3),(0,4),(1,0),
(2,1),(3,2),(0,3),(1,4),(2,0),
(3,1),(0,2),(1,3),(2,4),(3,0),
(0,1),(1,2),(2,3),(3,4),(0,0)
\end{verbatim}

It's clear that no two are the same, and since $r \times s = 20$, all possible pairs of remainders are found within this set.   

Let $a$ and $b$ be \emph{any} two positive integers.  Call their remainders \[ a' = a \text{ mod } r, \ \ \ \ b' = b \text{ mod } r \]

Then there exists an integer $N$ between $[1-20]$ such that

\[ N \equiv (a',b') \text{ mod } (r,s) \]

Furthermore, $N$ is uniquely determined.

In addition, there is a family of numbers $N + krs$ ($ k = 0, 1, 2 \dots$).  Every number in the family has the same remainder mod $r$ and, similarly, has the same remainder mod $s$, because $krs$ gives zero remainder with both $r$ and $s$.

\subsection*{example}
Let $r = 5$ and $s = 6$.  Suppose $a = 8$ and $b = 11$.
\[ a = 8 \equiv 3 \text{ mod } 5 \]
\[ b = 11 \equiv 5 \text{ mod } 6 \]
There must be some $N < rs$ (namely $23$), with the same remainders:  $N \equiv 3 \text{ mod } 5$ and $N \equiv 5 \text{ mod } 6$.  The next such number is $N + rs$.
\begin{verbatim}
 3  8 13 18 23 28 33 38 43 48 53
 5 11 17 23 29 35 41 47 53
\end{verbatim}


\end{document}  